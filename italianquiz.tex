\section{Intro}

Italianquiz is a web application that teaches Italian to English
speakers by displaying English sentences that must be translated to
Italian by the user. After the user submits a guess, the app returns
feedback by showing the user's guess along with the correct Italian
translation, along with feedback about how the user's response was
incorrect, if the app determines it to be incorrect. A Levenshtein distance
algorithm is used to highlight the parts of the guess that the users
guessed incorrectly, to draw the user's attention to these errors.

%% figure goes here.


\section{Background}

The app's linguistic formalism is a straightforward implementation of
HPSG (Head Driven Phrase Structure Grammar) by Pollum and Sag. In
particular:

- Linguistic expressions such as words, phrases and sentences are
represented as directed acyclic graphs (DAGs).

- Structure sharing within such graphs is used to model grammatical
concepts such as agreement and subcategorization.

- A distinction between phrasal heads and complements - heads select
their complements according to their subcategorization information.

- Lexemes encode their subcategorization information. This allows
phrase structure rules to be few and fairly abstract, because the
lexemes make the 

\section{Sentence Generation}

As explained in the Introduction, the app shows an English sentence to
the user and waits for a guess from the user of an Italian
translation. Behind the scenes, the app has already translated the
expression to Italian. When the user submits a guess, the app compares
this guess with its own translation, and to help the user improve,
highlights the diffrerence between the two translations. 

The sentences are generated by a head-driven, depth first process that
works as follows:

Given an input specification, the generator:

1. Determines the subset of the grammar that is possible given the
spec.

2. Choose a random rule R from this subset.

3. Choose from one of the following at random:

   - lexical head, lexical complement
   - lexical head, phrasal complement
   - phrasal head, phrasal complement
   - phrasal head, lexical complement




\section{Building a user's profile}

A user's errors can be used to guide app to match the user's
deficiencies. Because underlyingly, linguistics expressions are
feature-based, the app can build a profile of the user based on what
questions they answered incorrectly. Those path-values which
correlate with a high error rate indicate that the user needs to be
exposed to more of them in order to improve. For example, if a user
makes errors when conjugating the future tense, this can be correlated
with a high error rate for sentences with the path-value:

   {synsem {:sem {tense future}}}
 
\section{Generating felicitous sentences}

Generating high-quality sentences randomly requires care due to
semantic and cultural factors. Merely syntactic correctness is
insufficient: the app must avoid generating sentences like:

``The dog read the book'', or even worse:

``The book read the dog''. 

\section{Source Code}

Source code is available at: http://github.com/ekoontz/italianquiz and
is freely available to use under the terms of the GNU Public License.

\section{References}







